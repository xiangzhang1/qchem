\documentclass{article}
\usepackage{amsmath}
\usepackage{amsthm}
\usepackage{amssymb}
\usepackage{graphicx}
\usepackage{paralist}

%Partial Differential Symbols:\d{}{},\dd{}{},\pd{}{}
\let\underdot=\d % rename builtin command \d{} to \underdot{}
\renewcommand{\d}[2]{\frac{d #1}{d #2}} % for derivatives
\newcommand{\dd}[2]{\frac{d^2 #1}{d #2^2}} % for double derivatives
\newcommand{\pd}[2]{\frac{\partial #1}{\partial #2}}% for partial derivatives

%Quantum Mechanical Symbols:\ket{},\bra{},\braket{}{}
\newcommand{\ket}[1]{\left| #1 \right>} % for Dirac bras
\newcommand{\bra}[1]{\left< #1 \right|} % for Dirac kets
\newcommand{\braket}[2]{\left< #1 \vphantom{#2} \right|
 \left. #2 \vphantom{#1} \right>} % for Dirac brackets

%Normal Math:\abs{},\vec,\hf
\newcommand{\abs}[1]{\left| #1 \right|} % for absolute value
\newcommand{\hf}{\frac{1}{2}}

%Environments: prop,thm,lem,dfn & rmk
\newtheorem{prop}{命题}
\newtheorem{thm}{定理}[section]
\newtheorem{lem}[thm]{引理}
\theoremstyle{definition}
\newtheorem{dfn}{定义}
\theoremstyle{remark}
\newtheorem*{rmk}{注}

\usepackage{XeCJK}
\setCJKmainfont{SimSun}

\title{Axiomatization of Quantum Mechanics}
\author{Newcomer to QM}
\date{}

\begin{document}
\maketitle

\section{数学基础}

\begin{dfn}[希尔伯特空间]
  称完全的内积空间为\textbf{希尔伯特空间}。
\end{dfn}

\begin{dfn}[扩充希尔伯特空间]
  称内积空间$H$为扩充希尔伯特空间,若满足
  \begin{enumerate}[i)]
    \item 数域包括$\infty$
    \item 厄米算符本征矢完备正交(完全)
    \item 厄米算符完备组存在
  \end{enumerate}
\end{dfn}

\begin{dfn}[左右矢空间]
  称$L,R$为\textbf{左右矢空间},若
  \begin{enumerate}[i)]
    \item $R$为扩充希尔伯特空间
    \item $L,R$一一对应,以$\bra{\psi}\to\ket{\psi}$记之。
    \item $L,R$之间内积$\braket{\phi}{\psi}$满足共轭对称性、双线性性、正定性。
  \end{enumerate}
\end{dfn}

\begin{dfn}[算符]
  称$A$为\textbf{左右矢空间中的算符},若$A$满足性质
  \begin{enumerate}[i)]
    \item 右线性性
    \item $(\bra{\psi}A)\ket{\phi}=\bra{\psi}(A\ket{\phi})$
  \end{enumerate}
\end{dfn}

\begin{dfn}[伴算符]
  称$A^\dagger$为$A$的\textbf{伴算符},若
  \[\ket{\phi}=A\ket{\psi} \to \bra{\psi}=\bra{\psi}A^\dagger\]
\end{dfn}

\begin{dfn}[概率]
  称$P:\Omega\to\mathbb{R}$为概率,若:
  \begin{enumerate}
    \item $\forall A\subseteq\Omega,0\leq P(A)\leq 1$
    \item $A\bigcap B=\emptyset \to P(A\bigcup B)=P(A)+P(B)$
  \end{enumerate}
\end{dfn}

\begin{dfn}[随机变量]
  称$X:\Omega\to\mathbb{R}$为随机变量,并令$P(X\in A)=P(X^{-1}A)$。
\end{dfn}

\begin{dfn}[对易运算]
  定义对易运算
    \[[A,B]\triangleq AB-BA\]
\end{dfn}

\begin{rmk}
  有关对易运算的有关性质,参照\textrm{Wikipedia/Commutator/Ring}
\end{rmk}


\newpage
\section{物理模型}
\begin{dfn}
  称集合$\{H,\ket{\psi}(t),\{A\}\supseteq\{A_i\},\{r.v.(t) \hat{A}\}\}$为量子体系,若
  \begin{enumerate}
    \item[原理1] $H$为右矢空间,$\ket{\psi(t)}:H\to\mathbb{R}$,$\{A\}$为算符集,$\{\hat{A}\}$为随机变量集。存在以“粒子编号”$n$分组的位置算符$X_{ni}$,自旋角动量算符$S_{ni}$,以经典方式定义的正则动量算符$P_{ni},i=1,2,3$,以经典方式定义的哈密顿算符$H$,各算符列$A_n$。
    \item[原理2] 令$A\ket{a_i}=a_i\ket{a_i}$,$\ket{\psi}=\sum\ket{a_i}\braket{a_i}{\psi}$:
    \[\exists \{A\}\times H\to\{\hat{A}\}: P(\hat{A}=a_i)=\abs{\braket{a_i}{\psi}}^2\]
    \item[原理3] 位置算符$X_{ni}$,正则动量算符$P_{ni}$与自旋角动量算符满足
       \[[X_{ni},X_{nj}]=0,\qquad [P_{ni},P_{nj}]=0,\qquad [X_{ni},P_{nj}]=i\hbar\delta_{ij}\]
       \[\qquad [S_{ni},S_{nj}]=i\hbar\sum_{k}\epsilon_{ijk}S_{nk},\qquad [\vec{S_n},\vec{X_m}]=[\vec{S_n},\vec{P_m}]=0\]
    \item[原理4] 除测量时间以外,成立薛定谔方程
        \[i\hbar\frac{\partial}{\partial t}\ket{\psi(t)}=H\ket{\psi(t)}\]
    \item[原理5] 若$A_n=A_m$,则对$P_n\leftrightarrow P_m$,$\ket{\psi}\to\pm\ket{\psi}$
  \end{enumerate}
\end{dfn}
\vspace{\parskip}
\begin{dfn}
物理解释如下:
\begin{itemize}
  \item 概率是频率的极限,实验为测量该态下的物理量。
  \item 量子哈密顿量与经典哈密顿量以Bohm或Weyl或其他规则联系。以Bohm或Weyl或其他规则联系的量子物理量与经典物理量,测量方式相同。
\end{itemize}

\end{dfn}

\newpage
\section{基本定理}
\begin{thm}[不确定性关系]
  对算符$A,B$\footnote{不确定性关系是否只对量子力学成立?如果是,量子力学的概率结构与普通概率有何不同?}
  \[\triangle A\triangle B\geq\frac{1}{2}\abs{\overline{[A,B]}}\]
\end{thm}

\begin{thm}
位置与动量算符具有连续的本征值谱。
\end{thm}

\begin{dfn}[表象的函数形式]
  设位置算符$X$的本征矢量是完全的,单一算符构成完备组。定义矢量$\ket{\psi}$在表象下的函数形式$\psi_x$为如下函数:
  \[\psi_x\triangleq\braket{x}{\psi}\]
  定义算符$A$在表象下的函数形式$\hat{A}$为如下泛函:
  \[\ket{\phi}=A\ket{\psi}\to\phi_x=\hat{A}\psi_x\]
\end{dfn}

\begin{dfn}[表象的矩阵形式]
  设同上。将普通矩阵连续化得连续矩阵定义。定义矢量$\ket{\psi}$在表象下的矩阵形式$\psi_x$为如下连续矩阵:
  \[\psi_x\triangleq\braket{x}{\psi}\]
  定义算符$A$在表象下的矩阵形式$\hat{A}$为如下连续矩阵:
  \[\ket{\phi}=A\ket{\psi}\to\phi_x=\hat{A}\psi_x\]
\end{dfn}
\begin{rmk}
  以上定义合法。
\end{rmk}

\begin{prop}[球坐标可交换性]
  算符$R,\vec{N}$可交换。
\end{prop}

\begin{thm}[角动量算符本征值]
对任何角动量算符$J$,$[J^2,\vec{J}]=0$,记其一个共同本征矢组为$\ket{jm}$,则
\[J^2\ket{jm}=j(j+1)\hbar^2\ket{jm}\]
\[J_z\ket{jm}=m\hbar\ket{jm}\]
其中:$2j\in\mathbb{N}$,$m=-j.-j+1,\ldots,j$.
\end{thm}

\begin{thm}[$xyz$表象与$r\theta\phi$表象]
  当$X,Y,Z$与$R,\vec{N}$构成厄米算符完备组时,可直接分离定义$xyz$表象与$r\theta\phi$表象。此时$r\theta\phi$并非归一的。
\end{thm}

\begin{thm}[$rlm$表象与自旋表象]
当$R,L^2,L_z$构成厄米算符完备组时,$rlm$构成表象。自旋算符与位置算符共同构成厄米算符完备组,直积空间分别称为位置空间和自旋空间。
\end{thm}

\begin{thm}[单电子自旋空间]
  取$\ket{\hf,-\hf},\ket{\hf,\hf}$为自旋空间厄米算符完备组的公共本征矢。态矢量及算符在该表象下的矩阵形式为离散矩阵。令
  \[\vec{S}=\hf\hbar\vec{\sigma}\]
  $\vec{\sigma}$三分量在表象下的矩阵形式称为泡利矩阵。\\
  单电子希尔伯特空间同构于如下形式:
  \[\ket{\psi}=\ket{\psi_+}\ket{s_z+}+\ket{\psi_-}\ket{s_z-}\triangleq\binom{\ket{\psi_+}}{\ket{\psi_-}}\]
  最后哪一种是新的记法。
\end{thm}

\begin{thm}[一维谐振子]
  设$H$构成厄米算符完备组。通过基本的下降算符的技巧或在位置表象中计算可得其定态解$\ket{n},n=0,1,2,\ldots$
\end{thm}

\begin{thm}[氢原子与氢分子]
  采用位置表象与奥本海默近似,可计算出氢原子位置空间的定态解\footnote{为什么可以加上可分离变量的条件?}与氢分子的近似解。
\end{thm}

\begin{thm}[定态微扰法]
对于能量简并与不简并的情况,存在不同精度的定态解的近似。
\end{thm}


\begin{dfn}[海森伯绘景]
  对不含时的哈密顿$H^S$,定义海森伯绘景中的态矢量和算符
  \[\ket{\psi}^H=U^{-1}(t,0)\ket{\psi(t)}^S,\qquad A^H(t)=U^{-1}(t,0)A^SU(t,0)\]
  其中上标$S$表示薛定谔绘景即原态矢量,上标$H$表示海森伯绘景。$U(t_1,t_2)$为演化算符:
  \[U(t,t_0)\ket{\psi(t_0)}=\ket{\psi(t)}\]
  此时
  \[^H\bra{\psi}A^H\ket{\psi}^H=^S\bra{\psi}A^S\ket{\psi}^S\]
\end{dfn}

\begin{thm}
  薛定谔绘景中态矢量/算符的动K表象等于海森伯绘景中态矢量/算符的K表象.
\end{thm}

\begin{thm}[演化算符]
  在薛定谔绘景下,演化算符可以表示为级数解
  \[U(t,t_0)=1+\sum_{n=1}^{\infty}(-\frac{i}{\hbar})^n\int_{t_0}^t dt_1\int_{t_0}^{t_1} dt_2\cdots\int_{t_0}^{t_{n-1}} dt _n H(t_1)H(t_2)\cdots H(t_n)\]
  特别地,对于不含时哈密顿,演化算符为
  \[U(0,t)=e^{\frac{i}{\hbar}tH^S}\]
\end{thm}

\begin{thm}[海森伯绘景中的运动规律]
\[i\hbar\pd{}{t}\ket{\psi}^H=0\]
\[i\hbar\pd{}{t}A^H(t)=-[H,A^H(t)]\]
\end{thm}
\begin{thm}
  \[\frac{d}{dt}X^H_i(t)=\pd{H}{P^H_i(t)}\]
  \[\frac{d}{dt}P^H_i(t)=-\pd{H}{X^H_i(t)}\quad\footnote{对多项式$f(X)$,$[X,f(P)]=i\hbar\pd{}{P}f(P)$}\]
\end{thm}
\begin{rmk}
  这样一来,海森伯绘景倒是非常适用于量子化--参照经典系统运动规律写出其量子运动规律。
\end{rmk}

\begin{thm}[守恒量]
  若汇景中的算符不随时间变化,则称为守恒量。守恒量取各值的概率不随时间变化。
\end{thm}

\begin{thm}[相互作用绘景]
  若$H^S(t)=H_0^S+H_1^S(t)$,定义相互作用绘景下的态矢量$\ket{\psi(t)}^I$和算符$A^I(t)$:
  \[\ket{\psi(t)}^I=e^{\frac{i}{\hbar}tH_0^S}\ket{\psi(t)}^S\]
  \[A^I(t)=e^{\frac{i}{\hbar}tH_0^S}A_S(t)e^{-\frac{i}{\hbar}tH_0^S}\]
  则
  \[i\hbar\pd{}{t}\ket{\psi(t)}^I=H_1^I(t)\ket{\psi(t)}^I\]
  \[i\hbar\pd{}{t}A^I(t)=-[H_0^I,A^I(t)]\]
\end{thm}

\end{document}
