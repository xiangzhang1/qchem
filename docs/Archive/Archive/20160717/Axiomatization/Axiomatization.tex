\documentclass[11pt]{article}
\usepackage{amsmath}
\usepackage{amsthm}
\usepackage{amssymb}
\usepackage{graphicx}
\usepackage{paralist}
\usepackage{relsize}
\usepackage{fancyhdr}
\pagestyle{plain}

%\usepackage[BoldFont]{XeCJK}
%\setCJKmainfont[BoldFont=SimHei]{SimSun}
%\setCJKmonofont{SimSun}
\usepackage{ctex}


%Partial Differential Symbols:\d{}{},\dd{}{},\pd{}{}
\let\underdot=\d % rename builtin command \d{} to \underdot{}
\renewcommand{\d}[2]{\frac{d #1}{d #2}} % for derivatives
\newcommand{\dd}[2]{\frac{d^2 #1}{d #2^2}} % for double derivatives
\newcommand{\pd}[2]{\frac{\partial #1}{\partial #2}}% for partial derivatives

%Quantum Mechanical Symbols:\ket{},\bra{},\braket{}{}
\newcommand{\ket}[1]{\left| #1 \right>} % for Dirac bras
\newcommand{\bra}[1]{\left< #1 \right|} % for Dirac kets
\newcommand{\braket}[2]{\left< #1 \vphantom{#2} \right|
 \left. #2 \vphantom{#1} \right>} % for Dirac brackets
\newcommand{\langex}[1]{\overline{\{\mathcal{L}_{\text{#1}},\Gamma_{\text{#1}}\}}}%indicate an extended language
\newcommand{\lang}[1]{\{\mathcal{L}_{\text{#1}},\Gamma_{\text{#1}}\}}%indicate a language
\newcommand{\ptk}{\text{PTK}}%Indicate that a list of operators correspond to particles
\newcommand{\statsim}{\overset{S}{\sim}}

%Normal Math:\abs{},\vec,\hf,\nablabar
\newcommand{\abs}[1]{\left| #1 \right|} % for absolute value
\newcommand{\hf}{\frac{1}{2}}
\def\nablabar{\raisebox{-0.1ex}{--}\mkern-12mu \bigtriangledown}
\newcommand{\lims}{\text{lims}}

%Environments: prop,thm,lem,dfn & rmk
\theoremstyle{plain}
\newtheorem{thm}{定理}[section]
\newtheorem{dfn}{定义}[section]
\newtheorem{symb}{扩充符号}[section]
\newtheorem{axm}{扩充公理}[section]
\newtheorem{prop}{命题}[section]
\newtheorem{ppt}{性质}[section]
\theoremstyle{remark}
\newtheorem{rmk}{注}[section]


\title{公理化}
\author{张翔}

\usepackage{hyperref}

%混合态密度矩阵还没出来哎
%物理理论的一般形式,sim左边?
\begin{document}

\maketitle

\tableofcontents

\section{import}

在以下讨论中,我们遵循一些基本的规则,比如可以做什么、不可以做什么、把什么叫做什么。这种基本约定,叫做三段论。
还有一些语言不符合三段论,起文学解释作用,它们叫做说明。
还有一些肢体、直觉的动作,不见诸纸面,最终归结于\[\exists \alpha,\beta,\,\phi:\alpha\sim r.v. \text{A},\,\psi:\beta\sim r.v. \text{B}\]它们称为观测。

数学的讨论包括等价扩充与讨论、说明。
物理的讨论包括对物理体系的定义、等价扩充与讨论;特别地,包括验证:$\{\lang{},\phi\}\dashv\vdash\{\lang{},\psi\}$。物理的讨论还包括观测。

\section{公理集合论}

\noindent \textbf{以下基于三段论进行讨论。以下定义公理集合论的术语。}

\begin{dfn}[$\mathcal{L}$公式]
    由以下字母表中的若干有限个符号按照一定的规律形成的符号串叫\textbf{$\mathcal{L}$公式}。字母表包括:
    \begin{enumerate}
      \item 符号集$\neg,\land,\exists,\leftrightarrow,),(,=,\in$
      \item 变元集$x,y,z,\ldots$
      \item 谓词与函数词集$\mathcal{L}=\{\mathcal{L}_f,\mathcal{L}_p\}$
    \end{enumerate}
    规律为\\
    \begin{enumerate}
      \item $x\in y,x=y$是公式。
      \item 若$\phi,\psi$是公式,则\[\neg(\phi),(\psi)\land(\phi),\exists x(\phi)\]是公式。
      \item 公式中的$x,y$可换成其他变元或$f(x)$,其中$f$为函数词,$x$为变元。
    \end{enumerate}
\end{dfn}

\begin{dfn}
  称谓词与函数词集与其对应公式集$\lang{}'$为$\lang{}$的\textbf{扩张},记$\lang{}'=\langex{}$,若$\langex{}\supset\lang{}$
\end{dfn}

\begin{dfn}[等价扩张]
    设$\Gamma$为$\mathcal{L}$公式集,称谓词与函数词集与其对应公式集$\langex{}$为$\lang{}$的一个\textbf{等价扩张},若前者是后者经以下若干步形成:
    \begin{enumerate}
    \item $\mathcal{L}$中增加一个$n$元谓词符号$S$形成$\bar{\mathcal{L}}$,且$\Gamma$中增加一个新公理形成$\bar{\Gamma}$
      \[\forall x_1\cdots x_n(S(x_1,\cdots,x_n) \to \phi(x_1,\cdots,x_n))\]
      \item 公式$\phi(x_1,\cdots,x_n,y)$(其中列出了所有的可能的自由变元)满足
      \[\Gamma\vdash\forall x_1\cdots x \exists! y\phi(x_1,\cdots,x_n,y)\]
      且$\mathcal{L}$中增加一个$n$元函数词$f$形成$\bar{\mathcal{L}}$,且且$\Gamma$中增加一个新公理形成$\bar{\Gamma}$
      \[\forall x_1\cdots x_n\forall y(y=f(x_1,\cdots,x_n) \leftrightarrow \phi(x_1,\cdots,x_n,y))\]
\end{enumerate}
\end{dfn}

\begin{dfn}[$\vdash$]
    定义\textbf{$\vdash$}:设$\Gamma$是$\mathcal{L}$公式集,$\phi$是公式,称
    \[\lang{}\vdash\phi\]
    若存在着按一定规则写出的有限$\mathcal{L}$公式序列$\phi_1,\cdots,\phi_n$,使$\phi_n$就是公式$\phi$。规则为:
    \begin{enumerate}
        \item 在序列中若已有$\phi_i$,则可写出$\forall x\phi_i$,$x$为任一变元。
        \item 在序列中若已有$\phi_i$和$\phi_i\to\phi_j$,则后面可写出$\phi_j$。
        \item $\Gamma$的任—成员可在序列中任何地方写出。
        \item 序列中可随时写出以下形式的公式(叫做逻辑公理):
        \[\phi \to (\psi \to \phi)\]
        \[(\phi \to (\psi \to \chi)) \to ((\phi \to \psi) \to (\phi \to \chi))\]
        \[(\neg\phi \to \neg\psi) \to (\psi \to \phi)\]
        \[\forall x\phi(x) \to \phi(t),\quad\text{这里要求$t$可以自由代换$\phi(x)$中的$x$。}\]
        \[\forall x(\phi \to \psi) \to (\phi \to \forall x\psi),\quad\text{这里要求$x$不在$\phi$中自由出现}\]
        \item 序列中还可随时写出以下形式的公式(叫做等词公理):
        \[x=x\]
        \[x=y \to y=x\]
        \[( x=y \land y=z ) \to x=z\]
        \[x=y \to (\phi(x) \to \phi(y)),\quad\text{其中$\phi(y)$是用$y$替换$\phi(x)$一处或多处的$x$所得结果。}\]
        \item 以上允许公式中自由出现的变元可换成其他变元或$f(x)$,其中$f$为函数词,$x$为变元。
    \end{enumerate}
\end{dfn}


\begin{thm}[逻辑定理]
    若$\lang{}$的等价扩张$\langex{}\vdash\phi$,则$\lang{}\vdash\phi$。
\end{thm}

\begin{dfn}[$\lang{basic}$]
定义\textbf{$\lang{basic}$}
谓词与函数词集为
\[\mathcal{L}_{basic,f}=\emptyset,\mathcal{L}_{basic,p}=\{\forall,\land,\to,\leftrightarrow\}\]
基本公理集
\[(\phi \to \psi) \land (\psi \to \phi) \leftrightarrow (\phi\leftrightarrow\psi)\]
\[\neg((\neg(\neg(\phi)))\land(\neg(\psi)))\leftrightarrow(\phi)\to(\psi)\]
\[\neg((\neg(\phi))\land(\neg(\psi)))\leftrightarrow(\phi)\land(\psi)\]
\[\neg(\exists x(\neg(\phi)))\leftrightarrow\forall x(\phi)\]

\end{dfn}

\textbf{定义}是对$\lang{}$的等效扩充。

\section{概率论}

\noindent \textbf{定义$\lang{prob}$为$\lang{math}$的如下扩充:}

\begin{dfn}[概率]
  称$P:\Omega\to\mathbb{R}$为\textbf{概率函数},若:
  \begin{enumerate}
    \item $\forall A\subseteq\Omega,0\leq P(A)\leq 1$
    \item $A\bigcap B=\emptyset \to P(A\bigcup B)=P(A)+P(B)$
  \end{enumerate}
\end{dfn}

\begin{dfn}[随机变量]
  称$X:\Omega\to\mathbb{R}$为\textbf{随机变量},并令$P(X\in A)=P(X^{-1}A)$。
\end{dfn}

\begin{symb}
谓词$\sim$,函数词$\lims$.
\end{symb}
\begin{axm}
  若$A$为随机变量,\[{a_1,\ldots,a_n,\ldots}\sim{A}\to\lims\frac{\abs{x}}{\abs{y}}=\frac{P(\widehat{A}=x)}{P(\widehat{A}=y)}\]
\end{axm}
\begin{axm}
  $\lims$与$\lim$性质相同。
\end{axm}

\section{量子力学}

\subsection{数学基础}

\begin{dfn}[希尔伯特空间]
  称完全的内积空间为\textbf{希尔伯特空间}。
\end{dfn}

\begin{dfn}[扩充希尔伯特空间]
  称内积空间$H$为\textbf{扩充希尔伯特空间},若满足
  \begin{enumerate}[i)]
    \item 数域包括$\infty$
    \item 厄米算符本征矢完备正交(完全)
    \item 厄米算符完备组存在
  \end{enumerate}
\end{dfn}

\begin{dfn}[左右矢空间]
  称$L,R$为\textbf{左右矢空间},若
  \begin{enumerate}[i)]
    \item $R$为扩充希尔伯特空间
    \item $L,R$一一对应,以$\bra{\psi}\to\ket{\psi}$记之。
    \item $L,R$之间内积$\braket{\phi}{\psi}$满足共轭对称性、双线性性、正定性。
  \end{enumerate}
\end{dfn}

\begin{dfn}[算符]
  称$A$为\textbf{左右矢空间中的算符},若$A$满足性质
  \begin{enumerate}[i)]
    \item 右线性性
    \item $(\bra{\psi}A)\ket{\phi}=\bra{\psi}(A\ket{\phi})$
  \end{enumerate}
\end{dfn}

\begin{dfn}[伴算符]
  称$A^\dagger$为$A$的\textbf{伴算符},若
  \[\ket{\phi}=A\ket{\psi} \to \bra{\psi}=\bra{\psi}A^\dagger\]
\end{dfn}

\begin{dfn}[对易运算]
  定义对易运算
    \[[A,B]\triangleq AB-BA\]
\end{dfn}

\begin{rmk}
  有关对易运算的有关性质,参照\textrm{Wikipedia/Commutator/Ring}
\end{rmk}


\newpage
\subsection{量子力学}

\noindent \textbf{下面定义$\lang{qm}=\langex{prob}$,扩充如下:}

\begin{symb}
  谓词$\ptk$
\end{symb}

\begin{dfn}[量子体系]
  称集合$\{H,\{\ket{\psi}(t)\},\{A\},\{r.v.(t) \widehat{A}\}\}$为\textbf{量子体系},若
  \begin{enumerate}
    \item $H$为右矢空间,$\ket{\psi(t)}:\mathbb{R}\to H$,$\{A\}$为算符集,$\{\widehat{A}\}$为随机变量集,含位置算符$X_{ni}$,
    自旋角动量算符$S_{ni}$,以经典方式定义的正则动量算符$P_{ni},i=1,2,3,n=1,2,\ldots$,均$\ptk$,以及哈密顿算符$H$。
    \item 定义$A\ket{a_i}=a_i\ket{a_i}$,$\ket{\psi}=\sum\ket{a_i}\braket{a_i}{\psi}$:
    \[\exists \{A\}\times H\to\{\widehat{A}\}: P(\widehat{A}=a_i)=\abs{\braket{a_i}{\psi}}^2\]
    \item 位置算符$X_{ni}$,正则动量算符$P_{ni}$与自旋角动量算符满足
       \[[X_{ni},X_{nj}]=0,\qquad [P_{ni},P_{nj}]=0,\qquad [X_{ni},P_{nj}]=i\hbar\delta_{ij}\]
       \[\qquad [S_{ni},S_{nj}]=i\hbar\sum_{k}\epsilon_{ijk}S_{nk},\qquad [\vec{S_n},\vec{X_m}]=[\vec{S_n},\vec{P_m}]=0\]
    \item 除测量时间以外,成立薛定谔方程
        \[i\hbar\frac{\partial}{\partial t}\ket{\psi(t)}=H\ket{\psi(t)}\]
    \item 若$\forall \ptk{A_i}, A_n=A_m \land \braket{a_ni}{\psi}=\braket{a_mi}{\psi'}$,则$\ket{\psi}=\pm c\ket{\psi}$.
  \end{enumerate}
\end{dfn}

\newpage
\subsection{基本定理}

\begin{thm}[不确定性关系]
  对算符$A,B$\footnote{不确定性关系是否只对量子力学成立?如果是,量子力学的概率结构与普通概率有何不同?}
  \[\triangle A\triangle B\geq\frac{1}{2}\abs{\overline{[A,B]}}\]
\end{thm}

\begin{thm}
位置与动量算符具有连续的本征值谱。
\end{thm}

\begin{dfn}[表象的函数形式]
  设位置算符$X$的本征矢量是完全的,单一算符构成完备组。定义矢量$\ket{\psi}$在表象下的函数形式$\psi_x$为如下函数:
  \[\psi_x\triangleq\braket{x}{\psi}\]
  定义算符$A$在表象下的函数形式$\widehat{A}$为如下泛函:
  \[\ket{\phi}=A\ket{\psi}\to\phi_x=\widehat{A}\psi_x\]
\end{dfn}

\begin{dfn}[表象的矩阵形式]
  设同上。将普通矩阵连续化得连续矩阵定义。定义矢量$\ket{\psi}$在表象下的矩阵形式$\psi_x$为如下连续矩阵:
  \[\psi_x\triangleq\braket{x}{\psi}\]
  定义算符$A$在表象下的矩阵形式$\widehat{A}$为如下连续矩阵:
  \[\ket{\phi}=A\ket{\psi}\to\phi_x=\widehat{A}\psi_x\]
\end{dfn}
\begin{rmk}
  以上定义合法。
\end{rmk}

\begin{prop}[球坐标可交换性]
  算符$R,\vec{N}$可交换。
\end{prop}

\begin{thm}[角动量算符本征值]
对任何角动量算符$J$,$[J^2,\vec{J}]=0$,记其一个共同本征矢组为$\ket{jm}$,则
\[J^2\ket{jm}=j(j+1)\hbar^2\ket{jm}\]
\[J_z\ket{jm}=m\hbar\ket{jm}\]
其中:$2j\in\mathbb{N}$,$m=-j.-j+1,\ldots,j$.
\end{thm}

\begin{thm}[$xyz$表象与$r\theta\phi$表象]
  当$X,Y,Z$与$R,\vec{N}$构成厄米算符完备组时,可直接分离定义$xyz$表象与$r\theta\phi$表象。此时$r\theta\phi$并非归一的。
\end{thm}

\begin{thm}[$rlm$表象与自旋表象]
当$R,L^2,L_z$构成厄米算符完备组时,$rlm$构成表象。自旋算符与位置算符共同构成厄米算符完备组,直积空间分别称为位置空间和自旋空间。
\end{thm}

\begin{thm}[单电子自旋空间]
  取$\ket{\hf,-\hf},\ket{\hf,\hf}$为自旋空间厄米算符完备组的公共本征矢。态矢量及算符在该表象下的矩阵形式为离散矩阵。令
  \[\vec{S}=\hf\hbar\vec{\sigma}\]
  $\vec{\sigma}$三分量在表象下的矩阵形式称为泡利矩阵。\\
  单电子希尔伯特空间同构于如下形式:
  \[\ket{\psi}=\ket{\psi_+}\ket{s_z+}+\ket{\psi_-}\ket{s_z-}\triangleq\binom{\ket{\psi_+}}{\ket{\psi_-}}\]
  最后哪一种是新的记法。
\end{thm}

\begin{thm}[一维谐振子]
  设$H$构成厄米算符完备组。通过基本的下降算符的技巧或在位置表象中计算可得其定态解$\ket{n},n=0,1,2,\ldots$
\end{thm}

\begin{thm}[氢原子与氢分子]
  采用位置表象与奥本海默近似,可计算出氢原子位置空间的定态解\footnote{为什么可以加上可分离变量的条件?}与氢分子的近似解。
\end{thm}

\begin{thm}[定态微扰法]
对于能量简并与不简并的情况,存在不同精度的定态解的近似。
\end{thm}


\begin{dfn}[海森伯绘景]
  对不含时的哈密顿$H^S$,定义海森伯绘景中的态矢量和算符
  \[\ket{\psi}^H=U^{-1}(t,0)\ket{\psi(t)}^S,\qquad A^H(t)=U^{-1}(t,0)A^SU(t,0)\]
  其中上标$S$表示薛定谔绘景即原态矢量,上标$H$表示海森伯绘景。$U(t_1,t_2)$为演化算符:
  \[U(t,t_0)\ket{\psi(t_0)}=\ket{\psi(t)}\]
  此时
  \[^H\bra{\psi}A^H\ket{\psi}^H=^S\bra{\psi}A^S\ket{\psi}^S\]
\end{dfn}

\begin{thm}
  薛定谔绘景中态矢量/算符的动K表象等于海森伯绘景中态矢量/算符的K表象.
\end{thm}

\begin{thm}[演化算符]
  在薛定谔绘景下,演化算符可以表示为级数解
  \[U(t,t_0)=1+\sum_{n=1}^{\infty}(-\frac{i}{\hbar})^n\int_{t_0}^t dt_1\int_{t_0}^{t_1} dt_2\cdots\int_{t_0}^{t_{n-1}} dt _n H(t_1)H(t_2)\cdots H(t_n)\]
  特别地,对于不含时哈密顿,演化算符为
  \[U(0,t)=e^{\frac{i}{\hbar}tH^S}\]
\end{thm}

\begin{thm}[海森伯绘景中的运动规律]
\[i\hbar\pd{}{t}\ket{\psi}^H=0\]
\[i\hbar\pd{}{t}A^H(t)=-[H,A^H(t)]\]
\end{thm}
\begin{thm}
  \[\frac{d}{dt}X^H_i(t)=\pd{H}{P^H_i(t)}\]
  \[\frac{d}{dt}P^H_i(t)=-\pd{H}{X^H_i(t)}\quad\footnote{对多项式$f(X)$,$[X,f(P)]=i\hbar\pd{}{P}f(P)$}\]
\end{thm}
\begin{rmk}
  这样一来,海森伯绘景倒是非常适用于量子化--参照经典系统运动规律写出其量子运动规律。
\end{rmk}

\begin{thm}[守恒量]
  若汇景中的算符不随时间变化,则称为守恒量。守恒量取各值的概率不随时间变化。
\end{thm}

\begin{thm}[相互作用绘景]
  若$H^S(t)=H_0^S+H_1^S(t)$,定义相互作用绘景下的态矢量$\ket{\psi(t)}^I$和算符$A^I(t)$:
  \[\ket{\psi(t)}^I=e^{\frac{i}{\hbar}tH_0^S}\ket{\psi(t)}^S\]
  \[A^I(t)=e^{\frac{i}{\hbar}tH_0^S}A_S(t)e^{-\frac{i}{\hbar}tH_0^S}\]
  则
  \[i\hbar\pd{}{t}\ket{\psi(t)}^I=H_1^I(t)\ket{\psi(t)}^I\]
  \[i\hbar\pd{}{t}A^I(t)=-[H_0^I,A^I(t)]\]
\end{thm}


\section{力学}

\noindent \textbf{定义$\lang{Mech}=\langex{math}$,其扩充如下:}

\begin{symb}
  谓词$\sim$
\end{symb}
\begin{axm}
  \[\vec{\xi} \sim {m_i,q_i} \to \dot{\vec{\xi}}=\Omega\pd{H}{\xi}\]
  其中
  \[H=H_{\textrm{EM}}+H_G\]
\end{axm}


\section{统计力学}

\noindent \textbf{定义$\lang{StatMech}=\lang{Mech}\bigcup\langex{Prob}$,其扩充如下:}

\begin{dfn}[系综]
  若$x(t)\sim{m_i,q_i}$,则定义随机过程$X(t)$(称为\textbf{系综}):
  \[\dot{X(t)}=\Omega\pd{H}{x}(X)\]
  且对孤立系统有
  \[pdf f_X(x)=const \forall E(x)=E_0\]
\end{dfn}

\begin{thm}[各态遍历假设]
  若$x(t)\sim s.p.X(t)$,则
  \[\lims_t\overline{A(x)}=EA(X)\]
\end{thm}

定义$\lang{StatQM}=\lang{qm}\bigcup\langex{Prob}$,其扩充如下:

\begin{dfn}[量子系综]
  若$\ket{\psi(t)}\sim \textrm{QM System }A$,则定义随机过程$\ket{\Psi(t)}$(称为\textbf{量子系综}):
  \[H\ket{\Psi}=i\hbar\pd{}{t}\ket{\Psi}\]
\end{dfn}

\begin{thm}[混合态]
  若对$a_i$为$\ket{\psi}$的本征值,$\ket{\psi}(a_i)\sim s.p.\ket{\Psi}$,则
  \[\lims\overline{A}=p_i\bra{\psi_i}A\ket{n}\braket{n}{\psi_i}=\bra{n}(A\ket{\psi_i}p_i\bra{\psi_i})\ket{n}=tr A\rho\]
  此处对所有重复指标求和。
\end{thm}
\section{随机过程}
\noindent \textbf{下面对$\lang{Prob}$进行等价扩充与讨论}

\begin{dfn}[随机过程]
  [随机]函数$X(t):T\to\{r.v.X\}$又称随机过程。
\end{dfn}

\begin{dfn}[随机变量的收敛]
  称$X_t\overset{p}{\to}X, t\to t_0$当且仅当
  \[\lim_{t\to t_0}P(X_t=X)=1\]
\end{dfn}

\begin{dfn}
  称随机过程$X(t)$是可微随机过程,若
  \[\frac{X(t+\delta t)-X(t)}{\delta t}=X'(t),\delta t\to0\]
\end{dfn}

\section{电动力学}

\section{基本定义}
\noindent \textbf{定义电磁场$\vec E,\vec B$满足:}
\begin{eqnarray}
  \nabla \times \vec E=-\frac{\partial \vec B}{\partial t}\\
  \nabla \cdot \vec D=\rho_0\\
  \nabla \cdot \vec E=\frac{\rho_0+\rho'}{\epsilon_0}\\
  \vec D=\epsilon \vec E\\
  \nabla \cdot \vec B=0\\
  \nabla \times \vec H=\vec j_0+\frac{\partial \vec D}{\partial t}\\
  \nabla \times \vec B=\mu_0(\vec j_0+\vec j')\\
  \vec B=\mu \vec H
\end{eqnarray}\\
约定:\\
在问题的陈述中,若未指出$\delta$电荷密度,则有限正则条件$|\vec E|<+\infty$满足;若未指出无穷远处渐进性态,则无限远处正则条件$\vec E(+\infty)=0$满足。

\section{数学工具:$\nabla$}

\subsection{$\nabla$在不同微分基下的展开}
考虑正交坐标系$\vec r=\vec r(u,v,w)$,定义
\begin{align}
&dr^2=(L_udu)^2+(L_vdv)^2+(L+wdw)^2\\
&L_u\triangleq\sqrt{\frac{\partial x_i}{\partial u}^2},L_v\triangleq\sqrt{\frac{\partial x_i}{\partial v}^2},L_w\triangleq\sqrt{\frac{\partial x_i}{\partial w}^2}
\end{align}

考虑梯度
\begin{align}
  \nabla f&\triangleq \lim \frac{\iint_{\partial   (u:u+\triangle u v:v+\triangle v w:w+\triangle w )}\vec n\cdot f dS}{V}\\
  &=\lim \frac{\left(\iint_{u+\triangle u,v:v+\triangle v,w:w+\triangle w}-\iint_{u+\triangle u v:v+\triangle v w:w+\triangle w}\right) \vec e_u f L_vL_w dv dw+
  \cdots}{L_uL_vL_wdudvdw}\\
  &=\frac{1}{L_uL_vL_w}\cdot(\frac{\partial (\vec e_u f L_vL_w)}{\partial u}+\cdots)\\
  &=\frac{1}{L_uL_vL_w}\cdot(f\cdot\frac{\partial (\vec e_v\times\vec e_w L_vL_w)}{\partial u}+\vec e_u L_vL_w\frac{\partial f}{\partial u}+\cdots)\\
  &=\frac{\vec e_u}{L_u}\frac{\partial f}{\partial u}+\frac{\vec e_v}{L_v}\frac{\partial f}{\partial v}+\frac{\vec e_w}{L_w}\frac{\partial f}{\partial w}
\end{align}
或者

\section{符号运算法}
\begin{dfn}
  对任意线性函数$T$,
  \begin{equation}
  T(\nablabar)\triangleq\lim\frac{\int T(\vec n) dS}{V}.
  \end{equation}
\end{dfn}
用""括起来的整体都是$T(\nablabar)$的一部分。如无"",则整个项都是被""括起来的。


\begin{ppt}
  若$T=T_1$,则$T(\nablabar)=T_1(\nablabar)$.
\end{ppt}

\begin{ppt}[分解]
\begin{equation}
T(\nablabar,f_1,f_2)=T(\nablabar,f_{1c},f_2)+T(\nablabar,f_1,f_{2c})
\end{equation}
\end{ppt}

\begin{ppt}[拆解]
若
\begin{equation}
T(\nablabar)="T_1(\nablabar)f_{2c}"
\end{equation}
则
\begin{equation}
T(\nablabar)="T_1(\nablabar)"f_{2}
\end{equation}
\end{ppt}

\section{数制(部分)}
\begin{thm}
\[(a+b)_{\text{补}}=a_{\text{补}}\oplus b_{\text{补}}\]
其中$\oplus$为数值和。
\end{thm}

\section{单位制}
每一个单位都是一组指数或者说一组基加上指数


\begin{dfn}
\[\mathbb{U}={1,\ldots,N}\times\mathbb{N}\]
\end{dfn}
定义
\[\mathbb{RU}=\mathbb{R}\times\mathbb{U}\]
定义函数词$x[A]$
\[\forall x\in\mathbb{R}, x[A]={x,N}\in\mathbb{RU}\]
定义运算
$\forall a=x[A],b=y[A]$
\[a+b\triangleq(a+b)[A]\]
定义运算
$\forall a=x[A],b=y[B]$
\[ab\triangleq(ab)[A][B]\]

\section{数学分析(部分)}

\begin{dfn}[微分算符$d$]
    定义泛函$d:d[y(x)]=[dy](x,\delta x):$
    对$\delta y=A(x)\delta x+o(\delta x)$
    \[dy(x)=A(x)\delta x\]
\end{dfn}
\begin{dfn}[多重积分]
  定义\textbf{多重积分号}$\int f dV dt:$
  \[\int_{V\in A\\t\in B}f dV dt=\lim_{\delta\vec{X}\to 0}\sum_{A\times B}f\delta\vec{X}\]
\end{dfn}
\begin{dfn}[立体角]
  对锥$V$,定义\textbf{立体角}函数$\Omega{V}$:
  \[\Omega{V}=\int_{S:\text{Cover }V}\frac{dS}{R^2}\]
  积分
  \[\int f d\Omega=\lim_{S\to 0}\int f(P)\frac{dS}{R^2}\]
\end{dfn}


\end{document}
